\section{Overall Description}

\subsection{Product perspective}
Intro product perspective. Describes external interfaces: system,
user, hardware, software; 
operations and site adaptation, and
hardware constraints.

eMall helps to manage to limit the carbon footprint caused by our urban and sub-urban mobility needs,
providing a way to introduce minimal interference and constraints on our daily schedule.

\subsubsection{Entity diagram}
A general UML of the system here.

\subsubsection{State diagrams}
State diagrams here.

\subsubsection{Scenarios}
\begin{enumerate}[label=\textbf{\Alph*}.]
      \item \textbf{Registration} \\
            Einar is a driver of an electric vehicle that uses every day to go to his
            office. He decided to download the eMall app because he heard by a friend
            of him that he can discover all the charging points in the entire world,
            booking one, starting a charge and pay for the charge, entirely through
            the app. After having downloaded it, launches the app for the first time
            and select sign in button to register into the system. He provides all
            the personal data required to access in the system and accept to personalize
            his experience by selecting his car from a provided list of all the EVs.
            He submits his data and the system asks him to verify his account through email or phone number.
      \item \textbf{Book a charge} \\
            Edvar has the necessity to go shopping in the next days at the blue and yellow furniture retailer of the city. Knowing 
            that shopping will take some time he wants to find and book a CP nearby the shopping centre to charge his EV that uses every day.
            In the home of the app, that shows all the CP of the world in a map, he filtered the results on the location of the mall
            and the day he wants to go. After submitting the form, the home page will change according to his information and displays 
            a map of the selected zone with the available charging stations to book. He discovers that there is one available
            really close from the shopping centre. He selected the marker of the CP and are displyed the information about the charging point operator,
            the types of connectors, the availability a the actual moment of the research, the availability of them for the filtered date,
            the charging power at which the connector operates and the cost for recharging 1 kWh.
            To book the charge he has to select one connector that is available and indicate when he wants to start the charge
            and when to finish. The app, because Edvar accepted to insert his EV model, knows how much time is needed to charge his car so suggested
            the optimal time to book for a complete charge from 10\% to 100\%.
            He can accept the suggested book range or override the suggestion and modify the range at his willing inside the
            availability of the connector. He then confirms the booking of the charge and see the reservation on the reservations' tab.
      \item \textbf{Manage a charge} \\
            Anne planned a long trip from Oslo to Stockholm to do with her brand new EV. 
            Considering the suggested time by the app to do a full charge on her EV, she booked a three hour charge through the eMall 
            app at her trusted charging station with high power connectors.
            When she arrives at the CP station she parks in a free slot with the booked connector. Through the app she selects
            the reservation in the reservations' tab and starts the charge inside the app. The connector socket is unlocked and
            the charge can start by plugging the connector on the EV. Meanwhile waiting for the complete charge, Anne goes for a walk
            because she feels relaxed to control at any time the status and the remaining time of the charge with the app. When the charge
            is completed, as expected before the three hours, Anne is already back from the walk and a notification about the end of the charge 
            appears on Anne's phone, she disconnects the connector and gets back home to prepare the luggages for the trip. She doesn't worry about 
            the payment because is executed in background with the payment method that she indicates before the booking operation.
      \item \textbf{Charging point status} \\
            Erling is the owner of a restaurant and installed two charging columns in the parking slot in front of the restaurant because
            he wants to acquire good clients that can stop for charging the EV and have a meal at the restaurant. To make the CP accessible to the largest
            possible public he subscribes to the eMall-business for CPO app because he is interested in a service that permits to add and manage CP and make 
            the CP visible by EV drivers in the eMall app. After submitting the registration by providing essential information about the the company, including 
            VAT number of the restaurant and IBAN bank account to get payments from the driver he waits for the approval to be inserted into the app.
            When the approval arrives Erling inserts the charging point of the restaurant by specifying the number of sockets by type, the amount of power supplied by
            each socket and the API to connect the charging columns to the dashboard. With the dashboard he can visualize how many vechicles are charging in real time 
            and for each charging vehicle the amount of power absorbed and the time left to the end of the charge. He can visualize the import that gets from each 
            charge, decide the price for a charge and add special promotions to the charge to win the loyalty of the existing clients or acquire new clients.
\end{enumerate}

\subsection{Product functions}
A summary of the major functions offered by OUR system:

• User Registration \\
• User's car personalization  (user id)\\
• User search in the map\\
• User retrieve details of charging station (user id, id cp, date)\\
• User books a charge (user id, id cp, connector type, date, start and end reservation time)\\
• User starts a charge\\
• User retrieve details of STATUS of the charge\\
• User pays the charge \\
• User receive a notification of the end of the charging\\
• CPO's special offer ()\\
• CPO gives a different amount of power supplied by the socket, and monitor
the charging process to infer when the battery is full\\
• CPO retrieves details of status of a charging station (amount of energy available 
in its batteries, if any, number of vehicles being charged and, for each charging vehicle, amount of power
absorbed and time left to the end of the charge)\\
• CPO can acquire by the DSOs information about the current price of energy;\\
• CPO can decide from which DSO to acquire energy (if more than one is available);\\
• CPO can dynamically decide where to get energy for charging (station battery, DSO, or a mix thereof
according to availability and cost).\\


\subsection{User characteristics}
Who are the users?

\subsection{Assumptions, dependencies and constraints}
\subsubsection*{Domain Assumptions}
A list of domain assumptions, so fact that must be true in order the system to work.
\subsubsection*{Dependencies}
A list of API, OS, browser the app/webapp depends on
\subsubsection*{Constraints}
GDPR, encryption \dots?S
