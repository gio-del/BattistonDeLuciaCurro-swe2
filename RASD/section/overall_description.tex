\section{Overall Description}

\subsection{Product perspective}
Intro to product perspective. Describes external interfaces: system,
user, hardware, software; also
operations and site adaptation, and
hardware constraints.

\subsubsection{Entity diagram}
A general UML of the system here.

\subsubsection{State diagrams}
State diagrams here.

\subsubsection{Scenarios}
\begin{enumerate}[label=\textbf{\Alph*}.]
      \item \textbf{Registration} \\
            Einar is a driver of an electric vehicle that uses every day to go to his
            office. He decided to download the eMall app because he heard by a friend
            of him that he can discover all the charging points in the entire world,
            booking one, starting a charge and pay for the charge, entirely through
            the app. After having downloaded it, launches the app for the first time
            and select sign in button to register into the system. He provides all
            the personal data required to access in the system and accept to personalize
            his experience by selecting his car from a provided list of all the EVs.
            He submits his data and the system asks him to verify his account through email or phone number.
      \item \textbf{Book a charge} \\
            Edvar wants to plan a charge for the next days and uses the app to find and book a charging
            station nearby his office. In the home of the app he selected the book button and fills the
            information about the date and the time slot available for him to charge and indicates the
            zone in the map to scan for available charging stations. After submitting the form, the home page
            will change according to his information and displays a map of the selected zone with the available
            charging stations to book.

            Each charging station, if selected, shows the information about the charging point operator,
            the types of connectors, the availability of them for the filtered period,
            the charging power at which the connector operates and the cost for recharging 1 kWh.
            Edvar selected the charging station with the fastest connector but with the lowest price.
            To book the charge he selected one connector that is available and indicates when he want to start the charge
            and when to finish.
            The app, because Edvar accepted to insert his EV model, knows how much time is needed to charge his car so suggested
            the optimal time to book for a charge (from 10\% to 80\% of charge).
            He can accept the suggested book range or override the suggestion and modify the range at his willing inside the
            availability of the connector.
      \item \textbf{Other scenarios} \\
            text goes here
\end{enumerate}

\subsection{Product functions}
A summary of the major functions offered by the system

\subsection{User characteristics}
Who are the users?

\subsection{Assumptions, dependencies and constraints}
\subsubsection*{Domain Assumptions}
A list of domain assumptions, so fact that must be true in order the system to work.
\subsubsection*{Dependencies}
A list of API, OS, browser the app/webapp depends on
\subsubsection*{Constraints}
GDPR, encryption \dots?S
