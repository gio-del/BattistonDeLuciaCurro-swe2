\section{Introduction}\label{intro}
\subsection{Purpose}
With the higher focus on the impact of our urban and suburban travel on the environment and the higher accessibility of electric mobility, an increase of circulating electric vehicles can be observed.
\footnote{\url{https://www.eea.europa.eu/ims/new-registrations-of-electric-vehicles}} \footnote{\url{https://www.statista.com/statistics/1101415/number-of-electric-vehicles-by-type/}} \footnote{\href{https://www.eib.org/en/surveys/climate-survey/4th-climate-survey/hybrid-electric-petrol-cars-flying-holidays-climate.htm}{European Investment Bank Climate Survey}}
This increase concerns both private vehicles and goods transporting ones.
As a result of restrictions on fuel vehicle production and sell that will concern a large part of the world's population\footnote{\href{https://en.wikipedia.org/wiki/Phase-out\_of\_fossil\_fuel\_vehicles\#Places\_with\_planned\_fossil-fuel\_vehicle\_restrictions}{Places with planned fossil-fuel vehicle restrictions}}, the number of
electric vehicle is still set to increase. For these reasons, the main vehicle manufacturers have started making huge investments in electric mobility\footnote{\href{https://en.wikipedia.org/wiki/Electric\_car\#EV\_plans\_from\_major\_manufacturers}{EV plans from major manufacturers}}, which will lead to greater accessibility to the market by drivers.\\

A main problem of electric vehicles is that a full charge requires much more time than a fuel vehicle refuel.
\footnote{\href{https://blinkcharging.com/fact-from-fiction-the-real-reason-why-consumers-dont-buy-electric-vehicles/?locale=en}{Why consumers don't buy electric vehicles}}
Thus, a single charge can have a huge impact on our daily schedule, and it is necessary to plan wisely when and where to charge.
Furthermore, some electric vehicle owners don't have the proper equipment to recharge at home, or their vehicle discharges in the middle of the road and the driver doesn't have the possibility to go home to recharge.\\
To solve these problems is one of the main objective of the eMall - e-Mobility for All system.\\
This system aims to develop an efficient planning of the charging process of electric vehicles that limits the carbon footprint caused by people mobility needs.\\\\

The following document is the RASD for the eMall - e-Mobility for All system. It provides
a description of the system focusing on the requirements and specifications, developing scenarios and use cases
to specify what the system must do, how it will interact with the stakeholders and the constraints it is subject to.

\subsubsection{Goal}
\begin{table}[H]
    \begin{tabularx}{\textwidth}{cX}
        \toprule
        \textbf{G1} & Allow EV - Electric Vehicle driver to plan efficiently their charging process                   \\
        \textbf{G2} & Allow EV - Electric Vehicle driver to have a single application for all the processes involving
        the charge with a personalized experience based on the car and the user commitments                           \\
        \textbf{G3} & Allow CPOs - Charging Point Operators to be reached by EV drivers looking for charging points   \\
        \textbf{G4} & Provide smart managing of charging stations, including the register of reservations             \\
        \textbf{G5} & Allow CPOs - Charging Point Operators to choose between contracts of energy providers and
        to determine the energy source mix                                                                            \\ \bottomrule
    \end{tabularx}
\end{table}

\subsection{Scope}
The main actors in this system are the drivers and the CPOs - Charging Point Operators, who manage their charging columns, along with the DSOs - Distribution System Operator, in charge of distributing the energy.
The digital system eMall should provide three main features:
\begin{itemize}
    \item \textbf{Booking} allows EV owners to book a charge. The remote booking avoids interference
          in the daily schedule of the owners, and it includes a notification
          system that alerts owners when their reservation is going to start.
    \item \textbf{Charging} allows EV owners to charge an EV, remotely monitor their charging
          process and be notified at the end of the charge.
          Thanks to these features, owners have not anymore the need to
          physically go to the CP when they want to retrieve details of their charge.
    \item  \textbf{Managing an EVCP} allows CPOs to get statistics on live and historical details
          about their EVCP - Electric Vehicle Charging Pool, to acquire information about the current energy price by
          DSOs and to decide in an automated way where to get energy for charging.
\end{itemize}




\subsubsection{World phenomena}
\begin{table}[H]
    \begin{tabularx}{\textwidth}{cX}
        \toprule
        \textbf{WP1} & An EV driver arrives at a charging station             \\
        \textbf{WP2} & An EV driver wants to charge the car                   \\
        \textbf{WP3} & A charging station is connected to the electrical grid \\
        \textbf{WP4} & Some charging station has solar panels                 \\
        \textbf{WP5} & Some charging station has a storage battery            \\
        \textbf{WP6} & An EV battery discharges                               \\ \bottomrule
    \end{tabularx}
\end{table}
\subsubsection{Shared phenomena}
\begin{table}[H]
    \centering
    \begin{tabularx}{\textwidth}{c|X|c}
        \toprule
        ID            &                                                                                                                                                                       & Controlled by \\ \midrule
        \textbf{SP1}  & An EV driver books a charge at a certain charging station                                                                                                             & world         \\ \midrule
        \textbf{SP2}  & An EV driver search for a specific charging station                                                                                                                   & world         \\ \midrule
        \textbf{SP3}  & The system suggest to charge based on daily schedule, special offers and availability                                                                                 & machine       \\ \midrule
        \textbf{SP4}  & An EV driver starts the charging process                                                                                                                              & world         \\ \midrule
        \textbf{SP5}  & An EV driver receives a notification when the charging process is completed                                                                                           & machine       \\ \midrule
        \textbf{SP6}  & An EV driver pays for the charge                                                                                                                                      & world         \\ \midrule
        \textbf{SP7}  & The system shows to CPO the status of its charging station as amount of energy in batteries, number of vehicle being charged and for each the time left of the charge & machine       \\ \midrule
        \textbf{SP8}  & The system shows to CPO information about the DSOs                                                                                                                    & machine       \\ \midrule
        \textbf{SP9}  & A CPO decide to acquire energy from a certain DSO                                                                                                                     & world         \\ \midrule
        \textbf{SP10} & The system notifies an EV driver that the charging shift will begin shortly                                                                                           & machine       \\ \midrule
        \textbf{SP11} & An EV driver monitors the charging status                                                                                                                             & machine       \\ \midrule
        \textbf{SP12} & An EV driver deletes a reservation                                                                                                                                    & world         \\ \midrule
        \textbf{SP13} & A CPO decide to retrieve the historical reservations on its CPs                                                                                                       & world         \\ \midrule
        \textbf{SP14} & An EV driver retrieves the historical reservations                                                                                                                    & world         \\ \bottomrule
    \end{tabularx}
\end{table}

\subsection{Definitions, Acronyms, Abbreviations}

\subsubsection{Definitions}
\begin{itemize}
    \item EV Driver - Electric Vehicle Driver, people or entities who own an EV car and want to use the system for their charging needs
    \item EVCP - Electric Vehicle Charging Pool, is a station with multiple CPs
    \item CP - a synonym of EVSE - is a single charging column with multiple connectors
    \item Connectors - charging sockets which can be of different types (e.g. CCS2, Type2)
    \item OCPP - Open Charge Point Protocol \footnote{\href{https://www.openchargealliance.org/protocols/ocpp-201/}{OCPP Protocol}} - is a protocol that dictates the communication between CPMS and a controlled CP to achieve smart charging functionalities
    \item OCPI - Open Charge Point Interface \footnote{\href{https://evroaming.org/ocpi-background/}{OCPI Protocol}} - is a protocol that dictates the communication between CPMSs and eMsps to let the CPMSs to be accessible by multiple eMsps achieving roaming functionalities
\end{itemize}

\subsubsection{Acronyms}
\begin{table}[H]
    \begin{tabularx}{\textwidth}{cX}
        \toprule
        \textbf{RASD}  & Requirement Analysis and Specification Document \\
        \textbf{eMSP}  & Electric Mobility Service Provider              \\
        \textbf{EV}    & Electric Vehicle                                \\
        \textbf{CPO}   & Charging Point Operator                         \\
        \textbf{DSO}   & Distribution System Operator                    \\
        \textbf{CPMS}  & Charging Point Management System                \\
        \textbf{EVSE}  & Electric Vehicle Supply Equipment               \\
        \textbf{CP}    & Charging Point                                  \\
        \textbf{EVCP}  & Electric Vehicle Charging Pool                  \\
        \textbf{GPS}   & Global Positioning System                       \\
        \textbf{API}   & Application Programming Interface               \\
        \textbf{OCPP}  & Open Charge Point Protocol                      \\
        \textbf{OCPI}  & Open Charge Point Interface                     \\
        \textbf{OS}    & Operative System                                \\
        \textbf{VAT}   & Value-Added Tax (number)                        \\
        \textbf{IBAN}  & International Bank Account Number               \\
        \textbf{HTTPS} & HyperText Transfer Protocol Secure              \\
        \textbf{TLS}   & Transport Layer Security                        \\
        \bottomrule
    \end{tabularx}
\end{table}
\vspace*{1cm}
\subsubsection{Abbreviation}
\begin{table}[H]
    \begin{tabularx}{\textwidth}{cX}
        \toprule
        \textbf{WP$_x$}  & x-World Phenomena            \\
        \textbf{SP$_x$}  & x-Shared Phenomena           \\
        \textbf{G$_x$}   & x-Goal                       \\
        \textbf{D$_x$}   & x-Domain Assumption          \\
        \textbf{Dep$_x$} & x-Dependency                 \\
        \textbf{R$_x$}   & x-Functional Requirement     \\
        \textbf{U$_x$}   & x-Use Case                   \\
        \textbf{NFR$_x$} & x-Non Functional Requirement \\
        \bottomrule
    \end{tabularx}
\end{table}

\subsection{Revision history}

\subsection{Reference Documents}
Assignment document A.Y. 2022/2023 ("Requirement Engineering and Design Project: goal, schedule and rules")


\subsection{Document Structure}
This document is composed of six sections:
\begin{enumerate}
    \item  we introduce the problem and the goals of the system to be. In the scope subsection
          we provide a description of the various world and shared phenomena occurring. Lastly, we provide useful information to read this document properly, such as definitions and abbreviations
    \item  we provide an overall description of the system, along with a description of the users and their main functionalities. Moreover domain diagrams are presented and several scenarios are described, lastly we provide the domain assumption of the system to be
    \item  we specify the requirements of the system to be. This includes functional and non-functional requirements. Furthermore use cases diagrams are presented, along with a description of each use cases and a related sequence diagram. Lastly, we provide a mapping of the requirements on both goals and use cases
    \item  we provide a formal analysis of the system to be with Alloy
    \item  we provide an estimate of the effort spent by each group member
    \item  we list the used references
\end{enumerate}