\section{Introduction}
\subsection{Purpose}

The purpose of this document is to give a more detailed view of the eMall - e-Mobility for All - system presented in
the RASD, explaining architecture, components and their interaction, processes and algorithms that will satisfy the RASD requirements.
Additionally, it includes instructions regarding the implementation, integration and testing plan.
This document is intended to be a reference for the implementation of the system, and is aimed toward the developers, testers and project managers.

\subsection{Scope}
eMall is a system that allows EV - Electric Vehicle - Driver to plan efficiently their charging needs and CPO - Charging Point Operator - to be reached by EV Driver.
\\In particular, eMall allows EV Driver to search and then book a charge to a CP - Charging Point, at a specific
time, pay the charge, start a charge and being notified when the charging process is completed.
\\The system allows CPO - Charging Point Operator - to smartly manage their charging points, choosing rates and energy sources.
\\The system consists of two subsystems: eMSP - electric mobility service provider - and CPMSs - charging point management systems. The former offer functionality to drivers, the latter offer management options to operators. One of the focuses of this document is the description of how these entities interact with each other. Instead, it is out of the scope the design of the physical CPs - Charging Points that are assumed to be already implemented and tested.
\\The eMSP will interact with the CPMS of multiple CPOs
\\For a more detailed description of the features that the system offers to end users, please refer to the RASD.
\\
The architecture of the S2B is divided into three layers physically separated because installed on different tiers.
\\These layers are:
\begin{itemize}
    \item Presentation Layer: it manages the presentation logic and all the interactions with the end users
    \item Business Logic Layer: it manages the application functions that the S2B provide
    \item Data Layer: it manages the safe storage and the access to data
\end{itemize}
\subsection{Definitions, Acronyms, Abbreviations}
\subsubsection{Definitions}
\begin{itemize}
    \item Tier -  physical components or servers that make up a system
    \item Layer - a logical separation of components with related functionality
    \item S2B - System To Be, is the system we are designing
    \item EV Driver - Electric Vehicle Driver, people or entities who own an EV car and want to use the
          system for their charging needs
    \item EVCP - Electric Vehicle Charging Pool, is a station with multiple CPs
    \item CP - a synonym of EVSE - is a single charging column with multiple connectors
    \item Connector - charging socket that can be of different types (e.g. CCS2, Type2)
    \item Rate - the rate that the CPO decides to set for the CPs it manages. It contains a fixed part for
          parking and a variable part per kWh. Usually the rates are associated with a certain power (kW)
    \item OCPP - Open Charge Point Protocol \footnote{\href{https://www.openchargealliance.org/protocols/ocpp-201/}{OCPP Protocol}} - is a protocol that dictates the communication between CPMS and a controlled CP to achieve smart charging functionalities

\end{itemize}

\subsubsection{Acronyms}
\begin{table}[H]
    \begin{tabularx}{\textwidth}{cX}
        \toprule
        \textbf{eMall} & e-Mobility for All                              \\
        \textbf{RASD}  & Requirement Analysis and Specification Document \\
        \textbf{DD}    & Design Document                                 \\
        \textbf{EV}    & Electric Vehicle                                \\
        \textbf{CPO}   & Charging Point Operator                         \\
        \textbf{DSO}   & Distribution System Operator                    \\
        \textbf{CP}    & Charging Point                                  \\
        \textbf{EVCP}  & Electric Vehicle Charging Pool                  \\
        \textbf{EVSE}  & Electric Vehicle Supply Equipment               \\
        \textbf{CPMS}  & Charging Point Management System                \\
        \textbf{eMSP}  & Electric Mobility Service Provider              \\
        \textbf{SPA}   & Single Page Application                         \\
        \textbf{DBMS}  & Database Management System                      \\
        \textbf{CDN}   & Content Delivery Network                        \\
        \bottomrule
    \end{tabularx}
\end{table}
\vspace*{1cm}
\subsubsection{Abbreviation}
\begin{table}[H]
    \begin{tabularx}{\textwidth}{cX}
        \toprule
        \textbf{R$_x$} & x-Functional Requirement \\
        \bottomrule
    \end{tabularx}
\end{table}
\subsection{Revision history}
\subsection{Reference Documents}
\begin{itemize}
    \item Requirement Analysis and Specification Document (referred to as “RASD” in the document)
    \item Assignment document A.Y. 2022/2023 (”Requirement Engineering and Design Project: goal, schedule
          and rules”)
\end{itemize}
\subsection{Document Structure}

This document is composed of seven sections:
\begin{itemize}
    \item Introduction: This section provides an overview of the DD, including the scope of the project, definitions of key terms, references to other relevant documents, and an overview of the design
    \item Architectural Design: This section describes the high-level components and interactions of the system. It also includes a component view, a deployment view, a runtime view, and descriptions of selected architectural styles and patterns
    \item User Interface Design: This section outlines the design of the user interface (UI) of the system, including user experience (UX) flowcharts
    \item Requirements Traceability: This section provides a mapping between the requirements specified in the RASD and the components specified in the DD
    \item Implementation, Integration, and Test Plan: This section outlines the plan for implementing, integrating, and testing the system, including the order in which subsystems and components will be implemented
    \item Effort Spent: This section provides information on the effort spent on the design process
    \item References: This section includes a list of any references cited in the DD
\end{itemize}
