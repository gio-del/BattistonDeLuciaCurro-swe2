\section{Introduction}
\subsection{Purpose}

The purpose of this document is to give a more detailed view of the eMall - e-Mobility for All - system presented in
the RASD, explaining architecture, components and their interaction, processes and algorithms that will satisfy the RASD requirements.
Additionally, it includes instructions regarding the implementation, integration and testing plan.
This document is intended to be a reference for the implementation of the system, and is aimed toward the developers, testers and project managers.

\subsection{Scope}
eMall is a system that allows EV Driver to plan efficiently their charging needs and CPO - Charging Point Operator - to be reached by EV Driver.

In particular, eMall allows EV Driver to search and then book a charge to a CP - Charging Point, either at a specific
time, pay the charge, start a charge and being notified when the charging process is completed.

TODO: reviews the domain and product, summary of main architectural style/choices (e.g., 3-tier / 4-tier, …)
\subsection{Definitions, Acronyms, Abbreviations}
\subsubsection{Definitions}
\begin{itemize}
    \item EV Driver - Electric Vehicle Driver, people or entities who own an EV car and want to use the
          system for their charging needs
    \item EVCP - Electric Vehicle Charging Pool, is a station with multiple CPs
    \item CP - a synonym of EVSE - is a single charging column with multiple connectors
    \item  Connector - charging socket that can be of different types (e.g. CCS2, Type2)
\end{itemize}

\subsubsection{Acronyms}
\begin{table}[H]
    \begin{tabularx}{\textwidth}{cX}
        \toprule
        \textbf{eMall} & e-Mobility for All                              \\
        \textbf{RASD}  & Requirement Analysis and Specification Document \\
        \textbf{DD}    & Design Document                                 \\
        \textbf{EV}    & Electric Vehicle                                \\
        \textbf{CPO}   & Charging Point Operator                         \\
        \textbf{CP}    & Charging Point                                  \\
        \textbf{EVCP}  & Electric Vehicle Charging Pool                  \\
        \textbf{EVSE}  & Electric Vehicle Supply Equipment               \\
        \bottomrule
    \end{tabularx}
\end{table}
\vspace*{1cm}
\subsubsection{Abbreviation}
\begin{table}[H]
    \begin{tabularx}{\textwidth}{cX}
        \toprule
        \textbf{R$_x$} & x-Functional Requirement \\
        \bottomrule
    \end{tabularx}
\end{table}
\subsection{Revision history}
\subsection{Reference Documents}
Assignment document A.Y. 2022/2023 (”Requirement Engineering and Design Project: goal, schedule
and rules”)
\subsection{Document Structure}

This document is composed of seven sections:
\begin{itemize}
    \item Introduction: This section provides an overview of the DD, including the scope of the project, definitions of key terms, references to other relevant documents, and an overview of the design
    \item Architectural Design: This section describes the high-level components and interactions of the system. It also includes a component view, a deployment view, a runtime view, and descriptions of selected architectural styles and patterns
    \item User Interface Design: This section outlines the design of the user interface (UI) of the system, including user experience (UX) flowcharts
    \item Requirements Traceability: This section provides a mapping between the requirements specified in the RASD and the components specified in the DD
    \item Implementation, Integration, and Test Plan: This section outlines the plan for implementing, integrating, and testing the system, including the order in which subsystems and components will be implemented
    \item Effort Spent: This section provides information on the effort spent on the design process
    \item References: This section includes a list of any references cited in the DD
\end{itemize}
